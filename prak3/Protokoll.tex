\documentclass[10pt]{scrartcl}

\usepackage[utf8]{inputenc}
\usepackage{tabularx}
\usepackage[ngerman]{babel}
\usepackage[automark]{scrpage2}
\usepackage{amsmath,amssymb,amstext}
%\usepackage{mathtools}

\usepackage[]{enumerate}
\usepackage{graphicx}
\usepackage{lastpage}
\usepackage[perpage,para,symbol*]{footmisc}
\usepackage{listings}
\usepackage{color}
\usepackage{textcomp}
\definecolor{listinggray}{gray}{0.9}
\definecolor{lbcolor}{rgb}{0.9,0.9,0.9}
\lstset{
	backgroundcolor=\color{lbcolor},
	tabsize=4,
	rulecolor=,
	language=matlab,
        basicstyle=\scriptsize,
        upquote=true,
        aboveskip={1.5\baselineskip},
        columns=fixed,
        showstringspaces=false,
        extendedchars=true,
        breaklines=true,
        prebreak = \raisebox{0ex}[0ex][0ex]{\ensuremath{\hookleftarrow}},
        frame=single,
        showtabs=false,
        showspaces=false,
        showstringspaces=false,
        identifierstyle=\ttfamily,
        keywordstyle=\color[rgb]{0,0,1},
        commentstyle=\color[rgb]{0.133,0.545,0.133},
        stringstyle=\color[rgb]{0.627,0.126,0.941},
}
\usepackage[pdfborder={0 0 0},colorlinks=false]{hyperref}
\usepackage[numbers,square]{natbib}
\usepackage{float}

\lstset{numbers=left, numberstyle=\tiny, numbersep=5pt, breaklines=true, showstringspaces=false} 

%changehere
\def\titletext{Praktikum 3}
\def\titletextshort{Praktikum 3}
\author{Oliver Steenbuck, Karolina Bernat}

\title{\titletext}

%changehere Datum der Übung
\date{12.12.2012}

\pagestyle{scrheadings}
%changehere
\ihead{MT, Pareigis}
\ifoot{Generiert am:\\ \today}

\cfoot{Karolina Bernat, Oliver Steenbuck}


\ohead[]{\titletextshort}
\ofoot[]{{\thepage} / \pageref{LastPage}}

\setlength{\parindent}{0.0in}
\setlength{\parskip}{0.1in}

\begin{document}
\maketitle
\setcounter{tocdepth}{3}
\tableofcontents
\listoffigures
\lstlistoflistings


\section{Rakete}
	Der zu simulierende Raketen Flug besteht aus 3 teilen. Die zu simulierende Rakete aus 2 Stufen, die nacheinander die Rakete antreiben und danach abgeworfen werden.
	Es soll die Geschwindigkeit sowie die Höhe jeweils von Stufe 1 und Stufe 2 simuliert werden.
	

	\subsection{StufenGemeinsam} \label{sec:stufenGemeinsam}
	In dieser Phase wird die gesamte Rakete durch Stufe 1 beschleunigt.
	
	Die Masse der Rakete ergibt sich also aus.
	\begin{align} 
	m_{\text{Rakete}} = m_1 + m_2
	\end{align}

	Die Schubkraft der Rakete ergibt sich hier durch
	\begin{align}
	F_s = \text{Durchsatz}_1 \times \text{SchubProDurchsatz}
	\end{align}

	Die Erdanziehung die auf die Rakete wirkt kann durch berechnet werden.
	\begin{align}
	F_e = \frac{G * m_{\text{erde}} * m_{\text{Rakete}}}{r_{\text{erde}}^2}
	\end{align}
	
	Gegeben die oben berechneten Werten können wir nun die Beschleunigung der Rakete berechnen durch:
	\begin{align}
	a_{\text{Rakete}} = \frac{F_s - F_e}{m_{\text{Rakete}}}
	\end{align}
	
	\subsection{Stufe 2}
	In dieser Phase ist Stufe 1 ausgebrannt und beginnt zur Erde zurückzufallen während die Rakete nur noch aus Stufe 2 bestet die auch den Antrieb liefert. Beide Stufen sind hier also getrennt zu betrachten	
	
	\subsubsection{Stufe 1}
	
	\subsubsection{Stufe 2}
	Stufe 2 wird hier weiter als Rakete bezeichnet, somit ergeben sich unter Anpassung der Formeln aus \ref{sec:stufenGemeinsam} folgende neue Formeln zur Berechnung des Raketenfluges.
	
	Die Masse der Rakete besteht nur noch aus der Masse der zweiten Stufe.
	\begin{align} 
	m_{\text{Rakete}} = m_2
	\end{align}

	Die Schubkraft der Rakete ergibt sich jetzt durch die zweite Stufe.
	\begin{align}
	F_s = \text{Durchsatz}_2 \times \text{SchubProDurchsatz}
	\end{align}

	Die Erdanziehung die auf die Rakete wirkt kann durch berechnet werden.
	\begin{align}
	F_e = \frac{G * m_{\text{erde}} * m_{\text{Rakete}}}{r_{\text{erde}}^2}
	\end{align}
	
	\subsection{Antrieblos}
	Beide Stufen \texti{fliegen} jetzt antrieblos und damit nur noch unter Auswirkung der Erdanziehung.
	
	
	\subsection{•}	
	
\section{Tisch}	
		
\end{document}

