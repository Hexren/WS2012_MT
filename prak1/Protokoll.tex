\documentclass[10pt]{scrartcl}

\usepackage[utf8]{inputenc}
\usepackage{tabularx}
\usepackage[ngerman]{babel}
\usepackage[automark]{scrpage2}
\usepackage{amsmath,amssymb,amstext}
%\usepackage{mathtools}
\usepackage[]{color}
\usepackage[]{enumerate}
\usepackage{graphicx}
\usepackage{lastpage}
\usepackage[perpage,para,symbol*]{footmisc}
\usepackage{listings} 
\usepackage[pdfborder={0 0 0},colorlinks=false]{hyperref}
\usepackage[numbers,square]{natbib}

\lstset{numbers=left, numberstyle=\tiny, numbersep=5pt, breaklines=true, showstringspaces=false} 

%changehere
\def\titletext{Praktikum 1 : DGL}
\def\titletextshort{Praktikum 1}
\author{Oliver Steenbuck, Karolina Bernat}

\title{\titletext}

%changehere Datum der Übung
\date{31.10.2012}

\pagestyle{scrheadings}
%changehere
\ihead{MT, Pareigis}
\ifoot{Generiert am:\\ \today}

\cfoot{Karolina Bernat, Oliver Steenbuck}


\ohead[]{\titletextshort}
\ofoot[]{{\thepage} / \pageref{LastPage}}

\setlength{\parindent}{0.0in}
\setlength{\parskip}{0.1in}

\begin{document}
\maketitle
\setcounter{tocdepth}{3}
\tableofcontents
\listoffigures
\lstlistoflistings

\section{Van der Pol DGL}
	\subsection{Gleichung}
		\begin{align}
		&y(0) = 0\\
		&\dot{y}(0) = 1\\
		&\ddot{y} = 6 \cdot (1-y^2) \cdot \dot{y} -y
		\end{align}
	\subsection{Gleichung als DGL 1. Ordnung}
		\begin{align}
			&\dot{z} = 6 \cdot (1-y^2) \cdot z - y\\
			&\dot{y} = z
		\end{align}
		
	\subsection{Euler Verfahren}
	\begin{align}
		&z_{1_{n+1}} = z_{1_{n}} + h \cdot (6 \cdot (1-z_{2_{n}}^2) \cdot z_{1_{n}} - z_{2_{n}})\\
		&z_{2_{n+1}} = z_{2_n} + h * z_{1_n}
	\end{align}
	
	
	\subsection{Runge Kutta 2. Ordnung}
	Es gelte	
	\begin{align}
		g(t,y) = z \label{g}\\		
		f(y,z) = 6 \cdot (1-y^2) \cdot z - y \label{f}
	\end{align}
	Dann können wir durch einsetzen von (\ref{g}) und (\ref{f}) in Runge Kutta 2. Ordnung die Iterationsgleichungen erstellen:
	\begin{align}
		&y_{j+1}=y_j + \frac{h}{2} \cdot [g(t_j, y_j) + g(t_{j+1}, y_i h \cdot g(t_j, y_j))]\\
		&z_{j+1}=z_j + \frac{h}{2} \cdot [f(y_j, z_j) + f(y_{j+1}, z_j + h \cdot f(y_j, z_j))]
	\end{align} 
	
	
	
	

\end{document}

